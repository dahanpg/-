\documentclass[dvipdfmx,autodetect-engine]{jsreport}
\usepackage{tikz}

\usepackage{amsmath}
\usepackage{booktabs}
\usepackage{url}


\title{隠れマルコフモデルによるヒト腸内細菌叢の状態推定}
\author{浜田~研究室\\
\\
早稲田大学 先進理工学部\\
電気・情報生命工学科\\ 
バイオインフォマティクス研究\\
\\
江 涵\\
1Y15F043–1
}

\begin{document}

\maketitle

\begin{abstract}
本研究は、隠れマルコフモデルを用いてヒト腸内細菌叢の状態推定を行うことです。研究分野はメタゲノム解析として、

文章は随時更新されますので,\TeX のソースコードと合わせて,常に最新版を見るようにして下さい.また,この文章の内容に関するコメント等は大歓迎ですので遠慮なく言うようにして下さい\footnote{特に,誤りや誤植を発見した場合はすぐに教えて下さい.また\TeX のソースの記載の仕方に関しても,もっと良いものがあるかもしれません.そういうことも教えてもらえると私のためにもなります.この文章に記載する項目の追加も大歓迎します.}.

\vspace{5mm}

ここには論文の「概要」を記載します\footnote{この概要は,学会発表などで書く概要と同じようなものだと考えてもらってOKです.}.
副査の先生はとても忙しいので概要ぐらいしかちゃんと読めない場合もあります.
概要だけ読んでこの研究の重要性と自分の研究の内容が伝わるように記載するようにしてください.
概要で研究内容が面白そうであることがわかり,論文全体をパラパラ見たときにちゃんと書けていそうだとわかったら,中身までちゃんと読んでくれる可能性が高まります\footnote{厳しい言い方をすると,論文の体裁がめちゃくちゃですと,研究の中身もめちゃくちゃで信用できないと思われてもしょうがないです.}.

概要は論文本体の構成と同じく,研究背景(動機),研究手法,結果,結論の順に構造化をして記載するとわかりやすい概要になると思います.

\vspace{10mm}

\textbf{[浜田研の学生の皆さんへのメッセージ]}
ちゃんとした論文を書くということは,\textbf{皆さんが想像しているよりも非常に時間がかかる大変な作業}となります.常に,読み手側の立場に立ちながら,わかりやすくかつ不足の無い文章を書くことを強く意識するようにしてください.
一方で,これが身につけられていれば,職種に依らずに社会に出てから皆さんの大きな「強み」になることは間違いありません.皆さんには,卒論・修論の執筆を通してこの部分を習得し浜田研を巣立っていってもらいたいと切に願っております.大変だとは思いますが頑張って下さい.


\end{abstract}

\setcounter{tocdepth}{2} % subsectionまで目次に載せる
\tableofcontents

\listoffigures %図目次(これはなくてもよい)
\listoftables  %表目次(これはなくてもよい)

\chapter{研究背景}

この章では,適宜参考文献を参照しながら,皆さんの研究の研究背景・研究目的・研究動機等を記載します.書く際には以下の点に特に気を付けるようにしてください.
\begin{itemize}
\item 説明をすることなく専門用語(テクニカルターム)を使うことは避けて下さい.どうしても使わなくてはならない場合でも,(そのテクニカルタームについての定義・記載のある)参考文献を引用するようにして下さい.
\item  既存の研究に触れながら(必ず文献を参照すること),自分の研究の必要性・重要性を読者に伝えるようにして下さい.「なぜこの研究をやることが必要なのか?」ということを明確に記載するのが重要です.よくある悪い例としては,手法の提案の論文で,「そのような手法がない」ということだけを動機にしているものはダメです.
\item 卒論・修論の\textbf{研究背景が1ページに満たないのは論外}です.なぜこの研究を行う必要があるのかということを,背景や既存研究を丁寧に説明しながら記載すれば\underline{少なくとも}数ページ以上にはなるはずです.
\end{itemize}

\section{章・セクション・サブセクションを適切かつ有効に利用すること}

\begin{itemize}
\item 適切にセクション・サブセクションを使いながら文章を\textbf{構造化}して記載します.この構造は論文の読みやすさに直結しますので,構造を良く考えるようにして下さい.
\item \texttt{\textbackslash tableofcontents}をプリアンブルに記載することにより自動的に目次を作成できます\footnote{この文章の2ページ目にあるものです.通常の目次だけでなく,表目次や図目次も自動的に出力することができます.}.目次を見ながら全体の論文の構成を常に見直すようにしてください.
\item セクションやサブセクションの内部では,パラグラフ・ライティングを心がけて書くようにしてください.パラグラフ・ライティングに関する本を一冊以上読んでから執筆を開始することを強くお勧めします.例えば,\cite{倉島201211}や\cite{木下198109}などを読んでおくと良いと思います\footnote{浜田研にも常備されているはず.}.
\item これは必須ではありませんが,句読点は,全角カンマ`,'と全角ピリオド`.'に統一するとよいと思います(数式との親和性がよいためです).この際,例えば,`。'と`.'が混在してはいけません.
\item 最上位の構造として,章(chapter)を利用するか,セクション(section)を利用するかは決まりはありません.ただし,卒論や修論は,研究背景,手法などの各パートを通常の論文より詳しく書くのが原則です.そのため,\textbf{最上位部分には章を使い,その中でさらに構造化を行い詳細を記述するということを浜田研では推奨}したいと思います\footnote{第1章 研究背景として,半ページしかないのは気が引けますよね・・・?}.
\end{itemize}

\subsection{サブセクション}


\subsection{サブセクション}

\section{数式の書き方}

\begin{itemize}
\item 一般に数式は完全な一文ではないということを意識して下さい.数式単独で一つの文にするということは避けるようにしてください\footnote{これも文化による部分もあると思いますが,数学に近い本ほど,この原則が守られていると思います.もし,数式を1文にしたければ,少なくとも最後にピリオドが必要だと私は思っています.}.\\
%
\textbf{[正しい例] }
次に関数$f(x)$を
\begin{align*}
f(x) = \frac{a \sin(x)}{b \cos(x)}
\end{align*}
と定義する.ここで$a$と$b$は0でない定数である.\\
\textbf{[良くない例] }
次に関数$f(x)$を次のように定義する.
\begin{align*}
f(x) = \frac{a \sin(x)}{b \cos(x)}
\end{align*}
ここで$a$と$b$は0でない定数である.
\item 定義がされていない記号(ノーテーション)を利用することは絶対に避けてください(円周率$\pi$ぐらいまで一般的なものは説明なしでも問題ないかもしれませんが...).
\end{itemize}

\section{図・表の書き方}

図は図\ref{fig:example}のように記載します.
特に,図の説明の部分を良く読むようにして下さい\footnote{図の説明の書き方は文化によって異なる場合があります(図には説明は詳しく書かずにタイトルだけを記載する文化もあります).
浜田研では,我々の研究分野で広く使われる方式を採用します.皆さんが読む論文の大部分の図表は,説明が詳しく記載されているはずです.}.

\begin{figure}[h]
\vspace{50mm}
\caption{\label{fig:example}
図の例.ここの部分に図を読み取るために必要な情報を(基本的には)全て記載します.図とこの説明だけを見れば,(本文を一切見なくても)図を理解をすることができるというのが理想です.一方で,自分の主観をこの部分に書くべきではありません(例えば,精度が高い,早いなどは読み手によって変わってくるのでここで書くべきではありません.)
}
\end{figure}

同様に,表は表\ref{tab:example}のように記載します\footnote{\url{https://www.tablesgenerator.com/latex_tables}を使うと表のソースが作りやすいと思います.}.この例では,区切り線に\texttt{booktabs.sty}パッケージに定義されているものを使っていますが,通常の\texttt{hline}などを使っても大丈夫です.

\begin{table}[h]
\caption{\label{tab:example} 表の例}
\begin{center}
\begin{tabular}{lcc}
\toprule
A & B & C \\
\midrule
D & E & F \\
G & H & I \\
\bottomrule
\end{tabular}\\
{\footnotesize 
表の場合は下側に表を解釈するために必要な情報を不足なく記載する}
\end{center}
\end{table}

\textbf{すべての図と表は,本文中で必ず1度以上参照する}ようにして下さい\footnote{本文で一度も参照されない図表は存在意義がありません.}.

\section{参考文献の書き方}

\begin{itemize}
\item \textbf{参考文献の数に決まりはないですが,10未満は少なすぎる}と思います.もちろん,皆さんの研究が本当に斬新な研究で,他に参照する研究がないものであれば参考文献が少なくても問題ないですが...
\item 参考文献の記載には\texttt{bibtex}を利用すると良いと思います.
\texttt{bibtex}に関しては,例えば\url{http://www.fan.gr.jp/~ring/doc/bibtex.html}を参照して下さい.
%
\item bibtex用の文献フォーマットの作成は,\url{http://www.bioinformatics.org/texmed/}を使うと便利だと思います(PubMedに載っている論文の場合).
書籍の場合\url{http://lead.to/amazon/jp/}が便利です.
\end{itemize}



\section{その他,\TeX の書き方等に関して}

\begin{itemize}
\item 図\ref{fig:example}のソースのように,番号の参照は常に\texttt{\textbackslash ref}と\texttt{\textbackslash label}を利用すること.文献の参照は\texttt{\textbackslash cite}を使うこと.このような参照番号のベタ打ちは絶対に避けること\footnote{図表や文献が途中に追加された場合に番号がずれるためです.}.
\item その他\TeX の情報はHamadaLab Slackのtexチャンネルから得られると思います.不明点はここで質問してもいいかも.またウェブからも多数情報が得られるのでぜひググってみて下さい.
\item 論文のバックアップを定期的にとること.Overleafのサービスは完璧ではないので,サイトの不具合等によりデータが消失する可能性もあります.
\end{itemize}

以下宮原君から:

overleafについて,githubと連携させることでバックアップを取ることができます.(v2では動作を確認しています.)
浜田研のチームレポジトリに保存することも可能です.
\begin{enumerate}
\item 保存したい文書のページをoverleaf上で開く.
\item 左上のMenuボタンをクリック
\item SyncのGithubをクリック
\item Create a Github repositoryをクリック
以後,英語の指示に従っていけば作れるはずです.
(Ownerをhmdlabにすると,チームレポジトリに保存できます)
\end{enumerate}


\section{先輩の論文で参考になるもの}

例えば,皆さんの先輩が執筆した論文\cite{pmid29315213,pmid29040374}を見てみて下さい.英文の原著論文ですが,この文章で書かれていることが実践されていると思います.

また,今後は補遺\ref{chap:past_thesis}章に過去の論文のOverleafのリンクをまとめていくといいと思います(後輩が見ても恥ずかしくない論文の執筆を心がけましょう).

\chapter{手法}

研究手法について詳細を記載します.原則は,\textbf{論文の記載を読んで第3者が論文の結果を再現できるぐらいまで}詳しく不足なく記載をするということだと頭に入れておいて下さい.

\begin{itemize}
\item 論文は誰かに読んでもらうためのものです.常に,自分が読み手だった場合に今の記載の仕方で理解できるかという観点に立って執筆をするように心がけると良いと思います.
\end{itemize}

\chapter{結果}\label{sec:result}

本章には,結果を記載します.
\begin{itemize}
\item 結果と考察を分ける場合には,結果には得られた実験結果からわかる「客観的な事実」のみを記載するというのが原則となります.主観的な事柄に関しては考察に持っていくようにして下さい.ただし,これらを厳密に分けて書くことは難しいことも多く,「結果と考察」をまとめて一つの章とすることも行われます.
\end{itemize}

\chapter{考察}

ここでは考察を記載します.
\ref{sec:result}章では客観的な記載を心がけるように言いましたが,この章では結果から言える自分なりの解釈や主張を書いてもらって構いません.

\chapter{まとめ}

論文のまとめを記載します.

\chapter*{謝辞}

\begin{itemize}
\item 共同研究の場合はここでその旨を記載します.
\item スタッフ・先輩・同期・後輩などにいろいろと助けてもらった場合にはここに記載してもいいと思います.
\item 遺伝研のスパコンを利用した場合には以下の記載を行うようにしてください.
「本研究は,情報・システム研究機構 国立遺伝学研究所が有する遺伝研スーパーコンピュータシステムを利用しました.」
\end{itemize}

\bibliographystyle{unsrt}
\bibliography{ref}

\clearpage

\appendix 

\chapter{補遺}

補遺には,本文中に入れると分量が多く,論文の本筋がわかりにくくなる場合などに利用します.
\textbf{卒論・修論は,できるだけ自己完結的 (self-contained)に書くのが望ましい}ので,補遺を適切に利用し,可能な限り論文のみを読むだけで大部分が理解できるように記載することを心がけて下さい\footnote{逆に言うと,研究をするために勉強したことで,論文の内容を理解するためにあってもよいものはどんどん補遺に記載をしてもらって構いません.}.
\begin{itemize}
\item 既存の研究であるが,論文を参照せずに理解できるように手法を記載する.研究のために勉強した,基礎的なことであっても補遺に自分の言葉でまとめ直すことは全く問題ないです\footnote{このあたりは原著論文を書く際には少し違うので注意.}.
\item パラメタを少し変えたときの実験結果(表・図)などを載せる.
\end{itemize}


\chapter{卒論・修論執筆のスケジュール(現在検討中:来年度より本格運用予定.意見がある場合は私または学年代表まで)}

以下が卒論・修論執筆の大体のスケジュールになります.先に書いた通り,論文執筆には皆さんが想像する以上に時間・労力・精神力が必要となります.下記のスケジュールを\textbf{前倒しで実行}するぐらいの意気込みで,\textbf{常に先手先手でやるように}して下さい\footnote{精神的にもこのほうが楽です.}.

本スケジュールに関しては,学生側の意見も取り入れて随時見直しを行いたいと思いますので,遠慮なく意見を言うようにしてください.

\section{卒業論文}

% 来年度以降は下記のスケジュールでいくのがよいのではないかと思います.もちろん学生側から提案があれば,スケジュールに関しては相談に乗ります.

\subsection{Overleaf上に自分の卒論テンプレートの作成(夏前〜夏休み,早ければ早いほどよい)}

本文章のソースをコピーするということで構いませんので,なるべく早く作成をしてください.
作成後は放置するだけでは意味がないので,\TeX の書き方の練習も兼ねて,\textbf{研究に関連して勉強をしたことを随時 (例えば) 補遺にまとめていく}ことを強く推奨します.ここでまとめたものは,卒論の補遺や本文にそのまま残してもらっても構いません\footnote{文章が増えていくだけで心の余裕につながります.また,文章にまとめることにより,自分の中で研究の方向性が整理されるという効果もあると思います.}.

\subsection{卒業論文第一稿を教員に提出(12月の卒論正式提出の遅くとも1週間前)}

この時点の提出論文は正式提出1週間前のバージョンであり,それなりに完成度が高いものを期待しています.
そのため,第一稿の提出に先立ち以下を「必ず」行うようにして下さい.
\begin{itemize}
\item 第0稿を\textbf{卒研学生間での相互査読}をする.可能であれば,先輩にも見てもらう.
\item \textbf{論文の構成・内容に関して不明点があれば教員に事前に相談し解決しておく(重要!)}.想像してもらえればわかると思いますが,1週間前に提出したあとでは,大幅変更は難しいので,これは非常に重要です.
\end{itemize}

\subsection{卒研発表会と卒業論文の正式提出(12月上旬〜中旬)}

毎年12月中旬に,「卒研最終発表会」を情報系3研究室(井上,村田,浜田研)合同で行います.
この発表会の数日前に,卒業論文を3研究室で共有します.ここで提出する卒業論文は,村田先生や井上先生が読むことを強く意識し,その時点まで得られている結果に基づいたもので良いので,この文章の注意点を守ったちゃんとした(正式な)卒業論文を提出するようにして下さい.

\subsection{卒業論文の最終版提出(2月上旬)}

最終版提出までは,\textbf{12月の提出時点の正式版の卒業論文をさらに良くするための期間}です.
正式版提出後に新しく結果が出た場合などは論文に結果を追加してもらっても構いません.
さらに,日本語の卒業論文をベースにして,\textbf{この期間に「英文原著論文」の執筆に取り組めると素晴らしい}です\footnote{英文原著論の執筆は,mustではありませんが,ぜひ目標にしてもらえると良いかと思います.}.
そのためにも,12月時点で完成度が十分に高い論文を完成させておくことが重要です.

\section{修士論文}

卒論執筆で各自いろいろと反省点があると思いますので,それを修論に活かせるようにして下さい.

\subsection{Overleaf上に自分の修論テンプレートの作成(M1進学後すぐ)}

卒論以上に,\textbf{文章にまとめながら研究を行う}という習慣をつけるようにしてください.
2週間に一回のゼミでの進捗報告では,詳細はOverleafを参照するといった発表ができるとよいと思います.
文章にまとめることにより,研究の方向性が明らかとなったり,新しい気付きがあったりしますのでぜひ実践をしてください.

修論に関しては現状は中間発表等は一切行っていませんので,自ら強い意志を持って研究を進めていくことが必要になると思います.


\subsection{修士論文概要書の提出(M1の2月下旬ぐらい)}

A4で2ページの概要書の提出を行います.
\textbf{第0稿を学生間で相互査読を行った後に},遅くとも\textbf{専攻の提出締め切りの1週間前まで}に教員に第一稿の提出を行うようにしてください.

\subsection{修論の内容の学会発表(M2の夏ぐらい)}

\subsubsection{国内学会}

例えば,以下のような学会があります.
\begin{itemize}
\item 日本RNA学会
\item 日本バイオインフォマティクス学会(例年9月中旬〜下旬)
\item 分子生物学会(例年12月上旬に開催)
\end{itemize}

今までの学生は,日本バイオインフォマティクス学会で発表をしていた学生が多かったです.

\subsubsection{国際学会}

国際学会で発表を行う場合は,原則として論文が採択され口頭発表を行う場合となります\footnote{産総研RAとして雇用されている学生は,例外的にポスター発表での参加を認めることもあります.}.

\begin{itemize}
\item ISMB
\item GIW
\item APBC
\end{itemize}

\subsection{修士論文第一稿を教員に提出(下記正式提出の遅くとも1週間前,早ければ早いほどよい)}

この時点の提出論文は正式提出1週間前のバージョンであり,相当の完成度が求められています.第一稿の提出に先立ち以下を必ず行うようにしてください.

\begin{itemize}
\item 第0稿をM2学生間で相互査読をする.
\item 論文の構成・内容等に関しては,随時教員に相談を行い,不明点は解消しておく.
\end{itemize}


\subsection{修論の正式提出(M2の1月中旬〜下旬)}

学科に対して修士論文の正式提出を行います.
副査の先生が審査ができるように,修論発表会より1週間以上前に提出が義務付けられています.


\subsection{修士論文発表会(M2の2月上旬)}

\subsection{修論の最終提出(M2の2月中旬)}

\chapter{過去の浜田研学生の卒論・修論・博論}\label{chap:past_thesis}

今後(可能であれば今年度から)は,ここにOverleafのread-onlyリンクを残していきましょうかね・・・?

\section{2018年度}

\subsection{卒業論文}

\subsection{修士論文}

\subsection{その他の論文等}


\end{document}